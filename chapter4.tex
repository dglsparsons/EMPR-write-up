% Guidance - 2 pages
% Details of your team's testing strategy, and the results, including how well
% the solution met the requirements. Your evaluation should include a specific
% section relating to a professional/social/ethical/environmental aspect of
% your system solution.
\section{Description of the Team's Testing Strategy}

Throughout the duration of the project, my group was thorough and rigorous in our 
continual testing of the device through a variety of means. Through use of the 
version control software, git, new features were added as branches to the main
project. This enabled new features to be tested externally to the main body of 
the project, and furthermore enabled strict code reviews to be performed on each 
contribution to the group solution. 
In addition, a debugging library was implemented, allowing messages to be 
output for debugging purposes only, and easily removed from the solution. 
Furthermore, the group solution underwent extended testing and was continually 
running in the laborotories throughout the duration of the project. 

\subsection*{Individual Testing}

As our team used the version control system, git, we were able to produce new 
branches of the project for each new feature. 
This not only enabled each new features to be tested separately, but enabled 
different group members to work on alternate features simultaneously, with no 
possibility of interference. 
After the completion of a new feature, prior to the inclusion within the group 
solution, each section of the feature underwent individual tests. 
For a greater ease in testing the project,  a debugging library was created, 
enabling serial messages to print diagnostics. 
The debugging library was only included in the 
binaries installed on the device if the make command included it. This 
means, for the purposes of testing it was possible to use \texttt{make debug} 
and have messages containing text, or values printed out over serial. 
However, once sufficient testing was done, no 
alterations had to be made to the code to remove these statements, and 
instead the project could be installed by calling \texttt{make}. 
This therefore provided a simplistic tool, effectively creating two distinct 
modes of operation: a debugging mode, where information was output over serial, 
and a production mode, where normal operation ensued. 

\subsection*{Code Review}

Before each contribution was incorporated into the group's solution, it was 
first checked over by at least two other group members to ensure the quality of 
the solution was high. These checks often resulted in re-factoring code to 
achieve a more streamlined end product, examples of which could include the 
simplification of logic for writing text to the LCD display through pointer 
arithmetic, or simplifying code within an interrupt handler in order to avoid 
race conditions. 

\subsection*{Integration Testing}

Following the combination of individual components into the group solution, the 
combined product was then tested thoroughly to ensure no erroneous behaviour 
could occur. This was primarily done by leaving the solution running for long 
periods of time within the labs, and continually checking to ensure nothing unexpected 
had occurred at any point.
In addition, each new addition to the solution was fully tested by each group 
member on its incorporation within the project. 

\section{Results of Testing Strategy and How Well This Met the Requirements}

The groups testing strategy proved very efficient as capable. Through the use 
of individual branches for new features, any issues caused by new additions would 
not affect other parts of the group solution, and enabled a much simpler debugging 
process.  
Through the testing strategy many different issues were identified, and dealt 
with without before becoming significant probelm. 
Through leaving the group solution running for long periods of time, a large 
number of different test cases and situations were explored, such as the audio 
playback for many different tracks. This allowed us to ensure that the 
functionality of the end product was consistant.
Through our thorough and rigorous testing, the end product was remarkably stable. 
We encountered no situations that would cause the product to produce undesirable 
behaviour, and furthermore the full requirements were met.

\section{A Social Aspect of the Group Solution} 

The group solution presented has the potential to solve an issue that may arise 
when considering the usability of the device interface.
The host board has very limited interaction capabilities, with only the 16 digit 
keyboard providing a potential method for user input.
For many user's this method of interaction may not provide the most user friendly 
experience, and a lack of alternative modes of operation could cause  a 
significant issue.
Our solution to the issues that may arise from the lack of diversity of input 
methods comes through the incorporation of an alternative method of communication
with the device: the group solution implements interaction through a computer 
via a shell style interface.
Interaction with a computer, and hence the MBED, would typically be achieved 
through use of a keyboard, however, there are many alternative methods of input, 
such as speech recognition software. 
These alternative methods for user interaction with the device may provide a 
much more user friendly experience for many users. 
Methods of Human-Computer interaction for disabled users has been the study of 
a large number of individuals, such as Simpson and Stephanidis 
\cite{disabled-book, disabled-book2}. The improved usability of computer 
interaction provided through methods described in their works could extend to the 
MBED board through the group's implementation of multiple interface methods.
Hence, providing a solution to a potential social issue. 
