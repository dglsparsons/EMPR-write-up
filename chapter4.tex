% Guidance - 2 pages
% Details of your team's testing strategy, and the results, including how well
% the solution met the requirements. Your evaluation should include a specific
% section relating to a professional/social/ethical/environmental aspect of
% your system solution.
\section{Description of the Team's Testing Strategy}

Throughout the duration of the project, each section of the project underwent 
significant reviews, and testing in order to ensure that it was of the highest 
quality and that it was not laden with any issues causing undesirable behaviour, 
or significant faults that may have caused the device to crash. Furthermore, once 
incorporated into the group's main solution, the implemented code underwent a 
thorough review in order to ensure that the quality of the implemented product 
was the highest possible, and free from anything that may cause issues in the 
future. In effect the product underwent many different stages of testing 
throughout its life-cycle. 

\subsection*{Individual Testing}

Prior to the inclusion within a greater completed solution, each section of the 
project underwent individual tests. In order to test each section of the project, 
a debugging library was created. The debugging library was only included in the 
binaries installed on the device if the make command included included it. This 
means, for the purposes of testing it was possible to use \texttt{make debug} 
and have debug messages containing text, or values printed out over serial. 
However, once sufficient testing was done, no 
alterations had to be made to the source code to remove these statements, and 
instead the project could be installed by calling \texttt{make}. 
This enabled statements to be added to the project, that would print out 
messages over serial.

\subsection*{Code Review}

Before each contribution was incorporated into the group's solution, it was 
first checked over by at least two other group members to ensure the quality of 
the solution was high. These checks often resulted in re-factoring code to 
achieve a more streamlined end product, examples of which could include the 
simplification of logic for writing text to the LCD display through pointer 
arithmetic, or simplifying code within an interrupt handler in order to avoid 
race conditions. 

\subsection*{Integration Testing}

Following the combination of individual components into the group solution, the 
combined product was then tested thoroughly to ensure no erroneous behaviour 
could occur. This was primarily done by leaving the boards running for long 
periods of time within the labs, and listening to make sure nothing unexpected 
was occurring. In addition, each feature was continually explored in order to 
make sure everything fully functioned.  

\section{Results of Testing Strategy and How Well This Met the Requirements}

At many points throughout the project minor issues were discovered, and corrected. 
Through the continual testing of the full solution by leaving it running for a 
long period of time, a large number of test cases and situations were explored.
Furthermore, a careful eye was kept at all times to ensure that the product's
functionality was as desired, and further testing was done to ensure the 
requirements of the specification were fully met.
Through our thorough and rigorous testing, the end product was remarkably stable. 
We encountered no situations that would cause the product to produce undesirable 
behaviour, and furthermore the full requirements were met.

\section{A Social/Ethical Aspect of the Group Solution} 

The group solution created has the potential to solve any issues that may arise 
when considering the usability of the user interface with respect to a disabled 
user or users. Interaction with the device via the 16 digit keypad that is on 
the host board may not provide a very effective method of interaction with the 
device for many users. However, the shell style interface allows user to interact 
with the device via a computer. Interaction with a computer, and hence the mbed 
would typically be achieved through use of a keyboard, however, there are many 
alternative methods of input that would allow interaction with the mbed board in 
a method better suited for a disabled user. These input methods could include 
tools such as an improved computer keyboard, speech recognition, or the tailoring 
of custom tools to allow a greater level of interaction. Such tools are 
highlighted in many works, such as \textit{Computer Access for People with 
Disabilities: A Human Factors Approach}, or \textit{Universal Access in 
Human-Computer Interaction} [disabled-book, disabled-book2]. The flexibility of 
input method provides a wide range of possible tools enabling communication with 
the device. This therefore avoids significantly the problem of interaction for 
a handicapped or disabled user. 
