% Guidance - 2 pages
% Details of your team's testing strategy, and the results, including how well
% the solution met the requirements. Your evaluation should include a specific
% section relating to a professional/social/ethical/environmental aspect of
% your system solution.
\section{Description of the Team's Testing Strategy}

Throughout the duration of the project, each section of the project underwent 
significant reviews, and testing in order to ensure that it was of the highest 
quality and that it was not laden with any issues causing undesirable behaviour, 
or significant faults that may have caused the device to crash. Furthermore, once 
incorporated into the group's main solution, the implemented code underwent a 
thorough review in order to ensure that the quality of the implemented product 
was the highest possible, and free from anything that may cause issues in the 
future. In effect the product underwent many different stages of testing 
throughout its lifecycle. 

\subsection*{Individual Testing}

Prior to the inclusion within a greater completed solution, each section of the 
project underwent individual tests. In order to test each section of the project, 
a debugging library was created. The debugging library was only included in the 
binaries installed on the device if the make command included included it. This 
means, for the purposes of testing it was possible to use \texttt{make debug} 
and have debug messages containing text, or values printed out over serial. 
However, once sufficient testing was done, no 
alterations had to be made to the source code to remove these statements, and 
instead the project could be installed by calling \texttt{make}. 
This enabled statements to be added to the project, that would print out 
messages over serial.

\subsection*{Code Review}

Before each contribution was incorporated into the group's solution, it was 
first checked over by at least two other group members to ensure the quality of 
the solution was high. These checks often resulted in refactoring code to 
achieve a more streamlined end product, examples of which could include the 
simplification of logic for writing text to the LCD display through pointer 
arithmetic, or simplifying code within an interrupt handler in order to avoid 
race conditions. 

\subsection*{Integration Testing}

Following the combination of individual components into the group solution, the 
combined product was then tested thoroughly to ensure no erroneous behaviour 
could occur. This was primarily done by leaving the boards running for long 
periods of time within the labs, and listening to make sure nothing unexpected 
was occuring. In addition, each feature was continually explored in order to 
make sure everything fully functioned.  

\section{Results of Testing Strategy and How Well This Met the Requirements}

At many points throughout the project minor issues were discovered, and corrected. 
Through the continual testing of the full solution by leaving it running for a 
long period of time, a large number of test cases and situations were explored.
A careful eye was kept at all times to ensure that the product functioned as 
desired, and that all aspects of the requirements were covered. Through our 
thorough and rigorous testing, the end product was remarkably stable. 
No situations we encountered that would cause the product to display undesirable 
behaviour, and the testing strategy we incorporated rapidly caught any potential 
issues with the product. 

\section{Section relating to a professional/social/ethical/environmental
        aspect of our solution} % TODO - Find a proper name for this section
