% EMPR 2014/2015 Project report for hand in Week 3 Summer 2015

\documentclass[a4paper]{report}
\usepackage[pdftex]{graphicx}
\usepackage[margin=1.5in]{geometry}
\newcommand{\HRule}{\rule{\linewidth}{0.5mm}}

\begin{document}
\input{./title.tex}

\tableofcontents

\chapter{Introduction} 
% Guidance - 2 pages
% A summary of your project work, experiences as a team, expectations and
% actual outcomes, and some 'feel' for what the rest of your report is about
\section{Summary of project work}

\section{Experiences as a team}
Throughout the course of the project, many different challenges were undertaken
as a team.  
\section{Expectations and actual outcomes}
\section{What is in the report}
This report sets out to provide a description of our team solution to the set 
assignment, as well as highlighting areas of individual contribution throughout.
The report will begin with a technical description of the set problem, 
including a discussion of requirements and technical challenges that may be 
faced in the meeting of these requirements. 
Following a description of the problem, the group solution will be discussed. 
The discussion will highlight how each section of the requirements has been met 
by the implemented solution, as well as detailing how the problem has been broken 
down for each individual member, presenting each members implementation and 
technical innovation. 
After a discussion of the group's solution, the appropriate testing strategies 
and methods that were used throughout the project will be examined in detail, 
providing feedback on how they have proved useful, and incorporating a 
discussion on a professional/social/ethical/environmental aspect of the solution.
The report will then focus critically on my individual implementation, 
detailing the technical innovation, implemenation, and testing undergone for my 
contribution. 
Finally the report will conclude with a reflective summary of work 
undertaken, considering which aspects of the project went smoothly, which areas
did not go as smoothly, how this might be improved in future, and what lessons 
can be taken away from the completion of the project. 





\chapter{Technical Description of Problem} 
% Guidance - 3 pages
% A technical description of the problem in terms of the requirements given,
% expanded into a discussion and highlighting likely implied technical aspects
% and challenges that need/needed to be tackled
\section{Description of Problem and Requirements}
\section{Discussion of Technical Aspects and Challenges}

\chapter{Description and Discussion of Team-based Solution} 
% Guidance - 4 pages
% Description and discussion of the TEAM solution, noting how the team broke
% down the problem for team members and detailing the technical innovation and
% implementation of the work.
\section{Description of the Team Solution}
\section{How the problem was broken down for individual members}
\section{Technical innovation and implementation of each member}

\chapter{Evaluation and Testing of Team-based Solution} 
% Guidance - 2 pages
% Details of your team's testing strategy, and the results, including how well
% the solution met the requirements. Your evaluation should include a specific
% section relating to a professional/social/ethical/environmental aspect of
% your system solution.
\section{Description of the Team's Testing Strategy}
\section{Results of Testing Strategy and How Well This Met the Requirements}
\section{Section relating to a professional/social/ethical/environmental 
        aspect of our solution} % TODO - Find a proper name for this section

\chapter{Description, Discussion, Testing and Evaluation of 
            Individual Component}
% Guidance - 3 pages
% Description and discussion of your individual component solution, detailing
% the technical innovation and implementation and details of your testing
% strategy, and the results. 
\section{Description of Individual Component Solution}
\section{Discussion of Technical Innovation and Implementation}
\section{Testing Strategy, and the Results}

\chapter{Summary and Conclusions}
% Guidance - 1-2 pages 
% A reflective summary of the work undertaken (both team and individual),
% including thoughts about what went well, what did not, what could have been
% improved, and lessons learned. 
\section{Reflective Summary of Team Work}
\section{Reflective Summary of Individual Work}
\section{What went well}
\section{What went poorly}
\section{What could have been improved}
\section{Lessons Learned}

\chapter{Specified Documents}
% Guidance - As needed
% Evidence of project management (copies of meeting minutes), and evidence of
% preparation (copies of your Autumn formative assessment sheets)
\section{Meeting minutes}
\section{Evidence of Preparation}

\end{document}
