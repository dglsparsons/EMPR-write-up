% Guidance - 1-2 pages 
% A reflective summary of the work undertaken (both team and individual),
% including thoughts about what went well, what did not, what could have been
% improved, and lessons learned. 
\section{Reflective Summary of Team Work}

Team work was an essential part of the project, and an area where my group 
excelled. Each individual was allowed to select their own areas of the project 
to work on, and furthermore regular meetings and code review sessions both ensured 
that no individual was overwhelmed, and ensured that contributions to the 
solution were of the highest quality. 
\par\bigskip\noindent
The group solution achieved every point of the specification in an elegant 
manner, and had a wealth of additional functionality through additions to the 
project. 
Examples of these additions could include; the ability to synthesise music and 
play back multiple notes on each channel and multiple channels simultaneously, 
an improved user interaction via a shell style interface complete with the 
ability to adjust settings and change mode of operation without reinstalling the 
device, or a more realistic sounding audio playback through the use of attack-
delay-sustain-release models. 

\section{Reflective Summary of Individual Work}

My individual extension project was the implementation of a Linux style shell, 
enabling user interaction with the mbed board using a computer monitor and 
keyboard via serial communication. The addition of the shell style interface 
provided a wide range of valuable tools, and a great degree of control over the 
host board. Examples of the control given includes: 
\begin{itemize}
    \item The ability to play notes
    \item The ability to change the device volume
    \item the ability to write custom text to the LCD display 
    \item The ability to change which channel of music is being listened to
    \item The ability to view useful information, such as the track name, channel,
or what notes are currently being played
    \item The ability to view CAN packets as they are received on the CAN bus 
    \item The ability to scroll text on the LCD display 
\end{itemize}
This enabled a great degree of control over the host board, and provided 
invaluable tools throughout the course of the project. This extension project 
was initially very challenging and was an ambitious task to undertake due to the 
scale of the implementation combined with any desired features. However, the end 
result was very rewarding and one that I can be proud of. 

\section{What went well}


\section{What went poorly}


\section{What could have been improved}

Due to the nature of the project, there is a large amount of extension to the 
basic functionality that can be added. Our group solution extended the base 
implementation a long way, however, additional features were nearly implemented
that could have made a good improvement to the finished product. 
There are three examples of features that were not implemented in the final 
version of our project, but would have contributed to an overall improvement of 
the project. 
\par\bigskip\noindent
The first improvement would be the incorporation of multiple wave 
tables. 
Our group solution used sine waves to generate audio samples, however, 
there is scope for allowing any wave to be used. 
Through use of the shell it would be possible to alternate between different 
wave tables without the need to reinstall the binary on the device. This would 
allow a further degree of control over the device, and could be used to alternate
to more pleasurable sounding outputs for certain channels. 
\par\bigskip\noindent
The second potential improvement that could have been made ties in to the first, 
and is the implementation of drum samples to create a more complete sounding 
audio track. On the CAN bus sixteen channels of audio data is sent corresponding 
to the different instruments. Channel 11 typically represents a drum track, and 
for our project this channel is being ignored. This therefore is a potential
improvement that could have been made. 
\par\bigskip\noindent
The third potential improvement is a greater level of interaction between the 
shell and the keypad user input methods. As the keypad interaction was only 
implemented into the group solution near to the end of the project, the interfacing 
between the two methods is not flawless, and it is possible for minor 
discontinuities to occur. For example, if the user changes the volume using the 
keypad the new volume level is displayed on the LCD display. Changing the 
volume once again via the shell does not update the value displayed on the LCD 
display. The finished product could therefore be improved by improving the 
interactions between the two devices. 


\section{Lessons Learned}


