% Guidance - 1-2 pages 
% A reflective summary of the work undertaken (both team and individual),
% including thoughts about what went well, what did not, what could have been
% improved, and lessons learned. 
\section{Reflective Summary of Team Work}

Team work was an essential part of the project, and an area where my group 
excelled. Each individual was allowed to select their own areas of the project 
to work on, and furthermore regular meetings and code review sessions both ensured 
that no individual was overwhelmed, and ensured that contributions to the 
solution were of the highest quality. 
\par\bigskip\noindent
The group solution achieved every point of the specification in an elegant 
manner, and had a wealth of additional functionality through additions to the 
project. 
Examples of these additions could include; the ability to synthesise music and 
play back multiple notes on each channel and multiple channels simultaneously, 
an improved user interaction via a shell style interface complete with the 
ability to adjust settings and change mode of operation without re-installing the 
device, or a more realistic sounding audio playback through the use of attack
decay sustain release models. 

\section{Reflective Summary of Individual Work}

My individual extension project was the implementation of a Linux style shell, 
enabling user interaction with the MBED board using a computer monitor and 
keyboard via serial communication. The addition of the shell style interface 
provided a wide range of valuable tools, and a great degree of control over the 
host board. Examples of the control given includes: 
\begin{itemize}
    \item The ability to play notes
    \item The ability to change the device volume
    \item the ability to write custom text to the LCD display 
    \item The ability to change which channel of music is being listened to
    \item The ability to view useful information, such as the track name, channel,
or what notes are currently being played
    \item The ability to view CAN packets as they are received on the CAN bus 
    \item The ability to scroll text on the LCD display 
\end{itemize}
This enabled a great degree of control over the host board, and provided 
invaluable tools throughout the course of the project. This extension project 
was initially very challenging and was an ambitious task to undertake due to the 
scale of the implementation combined with any desired features. However, the end 
result was very rewarding and one that I can be proud of. 

\section{What went well}

My group had a very successful end product, and this was the result of many 
aspects of the project going very well. From the beginning of the project it was 
clear that each member of the group was very enthusiastic about producing a high 
quality end result, this enabled a positive working atmosphere where each group 
member was free to work on whatever aspect they desired as the project moved on.
\par\bigskip\noindent
Furthermore, we were aware of the limitations of the host MBED board prior to the 
start of the project, and therefore continually conscious of performance. Each 
feature implemented into the product was carefully considered, and its 
implementation was carefully monitored to ensure that it did not take excessive 
resources. Coupled with careful study of the data sheet for the host board, we 
were able to extract a great amount from the device, and produce a high quality 
finished product that was well optimised. 

\section{What went poorly}

While the vast majority of the project went very smoothly and according to plan, 
there were two occasions where significant issues were encountered. The first 
problem we encountered was in the first week of the project. After spending a 
week attempting to receive data from the CAN bus and failing, we decided to use 
the online tool-chain to set up a device to send CAN packets, and plugged this 
into our existing code. This worked perfectly, and led to the discovery that 
the CAN bus was not working everywhere within the lab. Because of this a lot of 
time within the first week of the project was spent attempting to correct code 
that was already working. 
\par\bigskip\noindent
The second issue that was faced was the inconsistent use of version control 
software. The majority of the group (Myself, Shivam and Liam) all used git, and 
github as a central location allowing us to effectively work as a team on the 
project. Mingzhao, despite being shown how to use git many times, elected not 
to use any version control software. This made interaction with his contributions
to the project much more difficult, and keeping up to date with his progress a 
much more difficult task. 

\section{What could have been improved}

Due to the nature of the project, there is a large amount of extension to the 
basic functionality that can be added. Our group solution extended the base 
implementation a long way, however, additional features were nearly implemented
that could have made a good improvement to the finished product. 
There are three examples of features that were not implemented in the final 
version of our project, but would have contributed to an overall improvement of 
the project. 
\par\bigskip\noindent
The first improvement would be the incorporation of multiple wave 
tables. 
Our group solution used sine waves to generate audio samples, however, 
there is scope for allowing any wave to be used. 
Through use of the shell it would be possible to alternate between different 
wave tables without the need to re-install the binary on the device. This would 
allow a further degree of control over the device, and could be used to alternate
to more pleasurable sounding outputs for certain channels. 
\par\bigskip\noindent
The second potential improvement that could have been made ties in to the first, 
and is the implementation of drum samples to create a more complete sounding 
audio track. On the CAN bus sixteen channels of audio data is sent corresponding 
to the different instruments. Channel 11 typically represents a drum track, and 
for our project this channel is being ignored. This therefore is a potential
improvement that could have been made. 
\par\bigskip\noindent
The third potential improvement is a greater level of interaction between the 
shell and the keypad user input methods. As the keypad interaction was only 
implemented into the group solution near to the end of the project, the interfacing 
between the two methods is not flawless, and it is possible for minor 
discontinuities to occur. For example, if the user changes the volume using the 
keypad the new volume level is displayed on the LCD display. Changing the 
volume once again via the shell does not update the value displayed on the LCD 
display. The finished product could therefore be improved by improving the 
interactions between the two devices. 

\section{Lessons Learned}

The project has served as a valuable learning experience throughout the duration, 
and has provided a great lesson to me personally in many different ways. 
First of all I have developed a much stronger understanding of the technical 
intricacies of programming in C - while the language itself is very minimal and 
straightforward to learn, there are many small details that require careful 
thought and consideration, such as the use of pointer arithmetic. Over ten weeks 
I have developed a greater knowledge and familiarity with this. 
Having no prior experience of programming an embedded system, this project has 
equally been an invaluable learning experience in that regard. 
Concepts such as interrupts, SysTick, or direct memory access are not entirely 
unfamiliar, however, the implementation desired features using these tools has 
been a new experience.
The project has also greatly developed my understanding of music technology. 
Through meeting the projects requirements to generate audio samples, and our own 
desire to make these samples sound more realistic a lot of research has been done 
into this field. 
Having no background in music technology it has been fascinating to research and 
learn about all the different aspects involved in our groups solution, such as 
sound synthesis, attack decay sustain release curves and equal loudness curves.
In addition, I have not worked on a group project before where version control 
software such as git has been used. Throughout the project I have gained a much 
better understanding of how a group should function, and I have developed a much 
better understanding of the communication between group members and how 
collaboration works. 
\par\bigskip\noindent
Overall there is a huge amount that I can take away from the Embedded Systems 
Project, and the project has provided a valuable opportunity to further develop 
my knowledge of embedded systems programming, working as a group, and music 
technology. 
