% Guidance - 4 pages
% Description and discussion of the TEAM solution, noting how the team broke
% down the problem for team members and detailing the technical innovation and
% implementation of the work.
\section{Description of the Team Solution}

The team solution following the completion of the project was a successful 
working product. Every point on the specification was fully met, and the solution 
incorporated additional features such as a user friendly shell style input, or 
the ability to synthesise multiple channels of music. This chapter sets out a 
full technical description of each different aspect of the finished product, 
explaining in detail how each feature is implemented. 

\subsection*{Controller Area Network Bus Processing}

The first technical challenge faced in the implementation of the group's solution
was the retrieval of data from the controller area network bus (CAN bus). The 
implemented solution uses an advanced system to ensure that no transmissions of 
data packets are missed. 
Each time data is transmitted on the CAN bus the MBED board receives an interrupt. 
The interrupt service routine copies the data transmitted on the CAN bus into 
the devices memory, ready for processing at a later stage. 
By avoiding the processing of any data in the interrupt handler, it is ensured 
that no race conditions can occur, and therefore data transmissions cannot be 
missed. 
\par\bigskip\noindent
The data packets are stored in memory as a queue. 
This enables a first-in-first-out approach to processing of the data to occur. 
The CAN data packets are decoded according to the given specification. 
There are two possible types of packets received: A basic data packet containing 
information for the frequency, volume, channel and control data of a note, or, 
a text packet, containing information for a track name, beats per minute of the 
track, and channel numbers and names \cite{data-packet}.
If the packet received is a data packet, it interacts directly with the audio 
code, turning on, off, or adjusting notes corresponding to the information 
contained. 
Alternatively, if the packet is a text packet, it requires further processing to 
extract the information contained. 
\par\bigskip\noindent
Text packets may arrive in multiple packets due to the length of text contained. 
Following the arrival of text packets, a check-sum is sent. 
If the length of the text received does not match the value of the check-sum, 
the text is ignored. 
This ensures all packets were successfully received, and that no packets have 
been corrupted. 
If the check-sum value is correct, and the text packets have been correctly 
received, they are parsed in order to extract their content into a more usable 
format. 
This is achieved through a state based text-parser, that draws inspiration from 
a stack-overflow post \cite{text-parser}. 
The text parser iterates through the text, alternating state depending on whether 
the current location is within a word, inside quotation marks or inside 
white-space. 
This breaks the text into smaller sections and removes quotation marks, 
enabling it to be presented in a useful format, such as displaying the track 
name on the LCD display, or on the computer screen. 

\subsection*{Shell User Interface}

To provide a user friendly and simplistic user interface, enabling a high level 
of control and interaction with the device, a shell style user interface was 
created. 
Typically used for interaction with a computer, a shell is a device that takes 
a user input, and converts their input (command) into the desired functionality. 
Simply put, a shell is "a program that takes your commands from the keyboard and 
gives them to the operating system to perform" \cite{shell-def}.
For the MBED system, communication with the device occurs via serial, with a 
computer running GNU screen (terminal) used for both input, and feedback from 
the device. 
This enables the user to enter commands via the computer keyboard, or other input
method, and the MBED device to then execute these commands. 
\par\bigskip\noindent
When a character is entered into the computer terminal, an interrupt is generated
on the host MBED board. 
The interrupt handler first determines whether the user's input is a special case 
character (BACKSPACE, or ENTER), or a regular character. 
If the input is a regular character, it is stored in memory for processing at a 
later stage, as well as being transmitted over serial to be displayed on the 
computer screen. 
If the input character is BACKSPACE the last character is removed from the 
stored input, and a special character sequence is sent to the computer in order 
to delete the displayed characters from the screen. This only occurs if the 
user has an input to delete 
Finally, if the user inputs the ENTER key, a flag is toggled on, indicating that 
the user's input requires processing. Processing of the user's input is not done 
in the interrupt handler in order to avoid unpredictable behaviour through race 
conditions.  
\par\bigskip\noindent
Processing of the user's input occurs in three different stages. 
First the input is parsed, where it is split up into separate words. This 
simplifies the logic needed to determine the correct functionality to perform.
Parsing is done using a similar method as the CAN bus text packets \cite{text-parser}, 
however, modifications have been made to set a limit on the word count of the 
user's input. This prevents the situation where an input consisting of a large 
number of separate words results in excessive memory consumption, causing the 
device to crash. 
Following the parsing of text, the user's input is matched to possible input 
commands using strcmp for string comparisons. This determines the correct 
functionality of the user's input. 
Once the correct functionality has been determined, the user's command is executed. 
Depending on what this command is, interaction may involve alterations being made 
to variables, states of the system, or may interface with the audio, or CAN bus 
segments of the group solution. 

\subsection*{Keypad and LCD interaction}

The keypad and the LCD additionally provide useful tools for interaction with 
the device. The specification states that the user must be able to display 
useful information on the host board, explicating the need to incorporate the 
LCD display into the solution \cite{specification}. 
The keypad was configured to work on an interrupt based system, as opposed to 
a poll-based system. This generated an interrupt each time a key was pressed. 
The interrupt handler first determined whether the user input was a valid key, 
filtering out any undesired interrupts generated from the row and column 
scanning of the keypad. 
Furthermore, a debounce system was generated, only allowing one successful 
interrupt to occur once every half a second. This was incorporated using SysTick
to determine the difference in time between inputs. 
Following a successful input, the correct functionality was then determined and 
executed, 
\par\bigskip\noindent
The LCD display was heavily utilised by the group solution, and had a wealth of 
features. 
The LCD code enabled writing to only a section of the display, allowing different 
information to be displayed on each line. 
This was achieved by manipulation of the device drivers, writing to only a 
section of the displays memory prior to updating the display. 
Furthermore the LCD display was further extended, adding the capability for 
scrolling text. 
This was implemented using pointer manipulation to incrementally iterate through 
a text string, and periodically update the display with this string using 
a system timer (SysTick).  

\subsection*{Audio Processing}

The audio processing aspect of the group solution was very thorough, and 
incorporated a vast array of different features to produce a high quality sound 
from the end product.
\par\bigskip\noindent
No information is known about the audio output, and wave-forms must be generated 
from first principles. This is implemented using a look-up table. 
In order to store a lookup table on the device, a tool was created to convert 
an audio sample into a list of values. This enables the possibility for alternate 
waves tables such as a piano sample to be incorporated into the solution, in 
addition to the more typical sine waves. This list of values was then stored in 
the flash memory on the device.
For each note, the wave-table is stepped over, and interpolated to provide readings 
between different values. 
The value at each moment in time then determines the output voltage. 
Only one wave table is used for the audio playback, different frequency notes are 
played back by adjusting the rate at which the wavetable is looped over. This is 
achieved my manipulating the step-size: stepping through the wavetable at a 
faster rate corresponds to playback of a higher frequency note. The step-size 
for each note is determined as each note is turned on by simply dividing the 
frequency of the note by the frequency of the wavetable. This method for the 
generation of notes is known as an oscillator \cite{puckette2007theory}.
\par\bigskip\noindent
In order to allow playback of more than one note at any one time, a method of 
audio synthesis, known as Additive Synthesis was implemented. This is a simplistic
method of representing multiple notes, and involves adding together their 
waveforms to create a composite wave \cite{miranda2012computer}. In the 
implemented solution, it is not possible to combine the wave tables together, 
and instead the values generated for each sample are instead added together. 
In order to create an output with consistant volume, the end value is also 
divided by the total number of notes that are playing at that moment in time. 
\par\bigskip\noindent
In order to enable each individual note to have a volume level, the 
value read from the wave-table is multiplied by a number between 0 and 1 
corresponding to the volume level of the note. 
However, the human ear does not perceive all frequencies equally, and further 
modification to the volume of each note has to be made in order to compensate 
for this. Equal volume loudness curves for different frequencies are described 
by the ISO 223:2003 standard. %TODO -- FIND A CITATION FOR THIS SPEC. 
Data points from this curve are stored in the devices flash memory, and 
interpolated to provide an accurate correction value for each frequency. This is 
applied to each individual note to produce a consistant sounding output. 
\par\bigskip\noindent



\section{How the problem was broken down for individual members}

The breakdown of our solution into components individual members could work on 
was a great strength of our team. Each person worked on an aspect of the project 
that they found interesting, and throughout the project we had regular feedback 
on all our work in order to make sure we were meeting our specifications and 
that nobody was out of their depth. Below I have divided up the work that each 
individual completed during the duration of the project, and why they were 
working on that aspect of the project. 

\subsection*{Mingzhao Zhao}
Each individual was allowed to choose their own area of the project to work on. 
Mingzhao was not especially confident with the platform, and therefore he 
decided to target user interaction with the device via the keypad. This 
provided a possible method of user input, as well as providing a means to 
optionally filter data, or control the volume output of the device. 
Following suggestions 
of other group members he decided to convert the previous keypad polling method 
incorporated in the mini-projects into an interrupt based system, removing the 
need for unnecessary CPU cycles to be used polling the device continually. 

\subsection*{Shivam Mistry} 
Shivam worked very strongly as a group member. To begin with he took on the 
challenge of constructing code to read the data from the CAN bus continually, 
as well as implementing the base for filtering out data according to a preset ID
. Following the construction of the CAN bus code, he worked on optimising the 
CAN bus code through the
implementation of a queue, as well as decoding the CAN packets to extract their
data, and the parsing of text packets to extract only the useful information. 
Furthermore, on realisation of additional device memory, Shivam wrote a memory 
allocator allowing the additional memory to be used constructively, further 
optimising the system. 

\subsection*{Myself (Douglas Parsons)}
Following the mini projects, and from reviewing the specification, it was clear 
that a strong level of user interaction with the device would provide valuable 
debugging tools, could potentially allow settings to be changed without 
re-installing binaries on the device, and could be used for a wide range of 
useful functions, such as changing the volume or displaying song information. 
From the mini projects it became clear that the 16 digit keypad attached to the 
MBED would not provide an intuitive user experience. Rather than relying on 
the keypad as the sole method of user interaction, I instead focused 
on allowing a further degree of interaction via the computer keypad. This 
enabled the user to preset an ID to filter out data, adjust the volume of the 
output, display any data received on the CAN bus, as 
well as adjust settings within the project without requiring a full reinstate.
During the implementation of the shell it became clear that this could be a 
valuable tool for displaying useful information such as the song name, and 
therefore I additionally worked with the LCD display to enable a much greater 
degree of control over what was displayed, enabling scrolling text, and writing 
to each line of the display individually.

\subsection*{Liam Fraser}
Liam has a strong background in music technology, taking the subject at A-level,
and working on personal projects involving music. Furthermore his interest from 
the beginning of the project in audio processing meant that it was an obvious 
choice for Liam to look into the generation and playback of music. The 
specification was exceeded with his contributions, not only did his solution 
generate an audio tone corresponding to the frequency, duration, and volume 
levels specified in the data stream, but implemented an attack sustain release 
curve to improve the realism of the output sound, as well as additive synthesis 
enabling multiple notes to be played simultaneously, and his solution also 
used DMA and a fixed point library to improve the efficiency of the system. 

\section{Technical innovation and implementation of each member}
\subsection*{Mingzhao Zhao}
Mingzhao's work was primarily focused on implementing interaction via 
the 16 digit keypad on the MBED board. The keypad was set up in the mini-projects 
to allow user interaction through continually polling the keypad. However, due 
to the limited speed of the processor it was decided that a better solution 
would be to use an interrupt based system.
However, this caused additional problems with bounce on the switches: 
multiple interrupts could occur in a very short space of time due to bounces on 
the switches. Many of these interrupts generated were redundant, or gave 
nonsensical input values, and furthermore their rapidity caused race conditions 
to occur within the interrupt handler. 
Due to the row/column method in which the key pad is scanned, the 
number of bounces was reduced significantly by filtering out any inputs that 
did not match one of the keys pressed. The switch bounces were then reduced 
further by only allowing a single key press every half a second. 
This method of debounce was implemented using a system 
timer (SysTick) to count the time between successive key presses. Any key 
presses that occurred too closely together were disregarded. 

\subsection*{Shivam Mistry}
Shivam initially targeted the CAN bus, and was able to rapidly achieve a fully 
functioning prototype. The original solution processed each packet as it was 
received. However, as the complexity of songs increased it became clear that this 
simplistic method was not going to be sufficient when many CAN packets require 
processing in a short space of time.
In order to improve the functionality, a queue was implemented. Packets were 
added to the queue on being received, and processed whenever there was sufficient
processor resources available.
This removed the requirement of each packet being processed as it was received,
 and therefore provided a clean, efficient solution.
Shivam's contribution also included the reverse engineering and decoding of any
 CAN packets received through testing their check-sum, and parsing any received 
text.
As the check-sum method was not specified in any documentation, the first 
challenge was determining what the check-sum corresponded to. Values that did 
not match the length specified by the check-sum were not processed, and therefore 
could not have any impact on the system. Furthermore, the text packets sent prior 
to each song being played required parsing in order to extract any useful 
information. Therefore, a text parser, extracting the useful information from 
these packets to the device was implemented using a state based solution. 
\par\bigskip\noindent
Furthermore, following an 
instance where the physical memory (RAM) on the device was fully used up, Shivam 
discovered that the device contains an additional 32Kb of memory, 
typically reserved for communicating to a USB or Ethernet device. However, 
due to the reserved status of this memory, a memory allocator was required for 
it to be utilised. 
Therefore, the group solution contains a memory allocator allowing access to 
a further 32Kb of memory. This allows a much greater amount of information to be 
stored on the device. 
In addition, the look-up table used by the memory allocator was stored in the 
additional memory, making it a self containing allocator. 

\subsection*{Myself (Douglas Parsons)}

My personal contribution to the project was primarily targeted at the user 
interactions with the device. While it was initially set out that Mingzhao was 
going to work on implementing an interrupt based keypad, I decided that the 
communication via serial would potentially allow a much greater level of 
debugging, interaction, and manipulation of settings for the device. From the 
beginning of the ten week period, I focused on setting up a shell style interface 
for the device. This would allow the user to communicate with the MBED board by 
typing in commands on a computer keyboard, and execute the command by pressing 
the enter key. The input from the keyboard was implemented using interrupts in 
order to avoid unnecessary computation time from continual polling. 
Furthermore, the processing of text input required a text-parser to be used. 
Due to an unknown number of words, and an unknown input length, the parser had 
to be very carefully adjusted to avoid memory overflow situations. Further 
difficulty was added to the implementation, as alterations had to be made in many 
different aspects of the group solution. A detailed understanding of each 
individuals contributions was required to manipulate their functionality, and 
continual additions and updates were needed throughout the project as changes 
were made. 
In addition to implementing the additional interfacing method of a shell style 
interface, I focused on customisation of the mini-project LCD screen code in 
order to allow a greater level of control. I implemented features such as the 
ability to write to each individual line of the LCD display, and the ability to 
scroll text across the screen. 
The latter of which was particularly difficult, not only requiring pointer 
manipulation for efficient text processing, but accessing only the required 
section of the LCD in order to avoid displaying unwanted characters provided 
a significant technical challenge. Following the implementation of the shell, and 
the LCD screen code, 
I worked alongside Mingzhao on combining his code with 
the rest of the group solution. 

\subsection*{Liam Fraser}
Liam Fraser, following an interest in music technology was eager to work on the 
audio code for the project. He implemented a simplistic system, allowing musical 
notes to be generated by scanning, and interpolating over a look up table. He then 
looked into generating synthesised music, and implemented additional synthesis 
into the project, allowing multiple notes to be playing simultaneously. However, 
on testing his synthesis code on the MBED board, it rapidly became apparent that 
the use of floating point numbers was not sufficiently quick to produce clean 
sounding audio. Therefore, a fixed point library was implemented, this allowed 
much quicker calculations to occur compared to those of floating point numbers. 
Following this optimisation of the synthesis code, the synth code was further 
optimised through the use of Direct Memory Access (DMA). This allowed the audio 
output to work independent of the central processing unit (CPU). This 
therefore speeds up the processing of any audio, as the CPU does not have to be 
involved for any sound to be output, only for the generation of audio samples 
\cite{dma-book}. In addition, to create a more realistic sounding output from the 
device, an attack sustain release state model was incorporated for each note. 
This gives an initial rise in volume as the note is turned on, and a fading out 
of the note as it is turned off, producing a volume curve more typical of an 
actual instrument \cite{asr-book}.

