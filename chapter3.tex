% Guidance - 4 pages
% Description and discussion of the TEAM solution, noting how the team broke
% down the problem for team members and detailing the technical innovation and
% implementation of the work.
\section{Description of the Team Solution}

The end product of our team solution was a successful working product. Through the 
incorporation of a Linux style shell into the group solution, we managed to produce 
a highly configurable single solution that managed to satisfy each section of 
specification. In addition the group solution contained two custom built speakers, 
capable of being powered by the 5V voltage rails from the peripherals board. 
The solution had a strong user interface, allowing the user to type in commands 
through the computer keyboard to adjust settings and select modes, set an id 
to filter channels out, and display information, or custom text on the mbed 
board. In addition it was also possible to adjust volume, channel and display 
information on the mbed board using the 16 digit keypad. This high level of 
user interfacing allowed settings to be flexibly turned on and off, and 
avoided the need to be continually re-installing binaries to the device. 
By typing in the command "listen", the mbed would then initialise the CAN bus 
and begin receiving packets. These could be printed out to the screen by issuing
a further command, or they could be used to generate audio tones. It is also 
possible to filter out data according to a preset id by issuing a command, or 
using the keypad. For each data element received, audio tones can be generated. 
The solution supports a full synthesis, allowing multiple notes to be played
on each channel, and multiple channels to play notes simultaneously. In addition 
the audio output also implemented an 'attack-sustain-release' model, creating a 
more realistic sounding solution. The audio code used direct memory access (DMA)
to provide a reduced processing cost in outputting the audio samples.
The user could also choose to display useful information such as the current 
song name, the volume, or the channel names on the computer screen or on the 
MBED board in a variety of ways. For example they could be statically printed on
the MBED board, or scrolled across the LCD display, or even printed out to 
the terminal on the computer.

\section{How the problem was broken down for individual members}

The breakdown of our solution into components individual members could work on 
was a great strength of our team. Each person worked on as aspect of the project 
that they found interesting, and throughout the project we had regular feedback 
on all our work in order to make sure we were meeting our specifications and 
that nobody was out of their depth. Below I have divided up the work that each 
individual completed during the duration of the project.

\subsection*{Mingzhao Zhao}
Mingzhao's work was primarily focused on handling the 16 digit keypad on the 
mbed board, this board was configured to be interrupt based rather than being 
polled. This caused additinal problems with bouncy switches sending multiple 
interrupts and race conditions to occur within the interrupt handler.

\subsection*{Shivam Mistry}

\subsection*{Myself (Douglas Parsons)}

\subsection*{Liam Fraser}

\section{Technical innovation and implementation of each member}

\subsection*{Mingzhao Zhao}

\subsection*{Shivam Mistry}

\subsection*{Myself (Douglas Parsons)}

\subsection*{Liam Fraser}
