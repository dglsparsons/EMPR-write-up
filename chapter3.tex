% Guidance - 4 pages
% Description and discussion of the TEAM solution, noting how the team broke
% down the problem for team members and detailing the technical innovation and
% implementation of the work.
\section{Description of the Team Solution}

The end product of our team solution was a successful working product. Through the 
incorporation of a Linux style shell into the group solution, we managed to produce 
a highly configurable single solution that managed to satisfy each section of 
specification. In addition the group solution contained two custom built speakers, 
capable of being powered by the 5V voltage rails from the peripherals board. 
The solution had a strong user interface, allowing the user to type in commands 
through the computer keyboard to adjust settings and select modes, set an id 
to filter channels out, and display information, or custom text on the mbed 
board. In addition it was also possible to adjust volume, channel and display 
information on the mbed board using the 16 digit keypad. This high level of 
user interfacing allowed settings to be flexibly turned on and off, and 
avoided the need to be continually re-installing binaries to the device. 
By typing in the command "listen", the mbed would then initialise the CAN bus 
and begin receiving packets. These could be printed out to the screen by issuing
a further command, or they could be used to generate audio tones. It is also 
possible to filter out data according to a preset id by issuing a command, or 
using the keypad. For each data element received, audio tones can be generated. 
The solution supports a full synthesis, allowing multiple notes to be played
on each channel, and multiple channels to play notes simultaneously. In addition 
the audio output also implemented an 'attack-sustain-release' model, creating a 
more realistic sounding solution. The audio code used direct memory access (DMA)
to provide a reduced processing cost in outputting the audio samples.
The user could also choose to display useful information such as the current 
song name, the volume, or the channel names on the computer screen or on the 
MBED board in a variety of ways. For example they could be statically printed on
the MBED board, or scrolled across the LCD display, or even printed out to 
the terminal on the computer.

\section{How the problem was broken down for individual members}

The breakdown of our solution into components individual members could work on 
was a great strength of our team. Each person worked on as aspect of the project 
that they found interesting, and throughout the project we had regular feedback 
on all our work in order to make sure we were meeting our specifications and 
that nobody was out of their depth. Below I have divided up the work that each 
individual completed during the duration of the project, and why they were 
working on that aspect of the project. 

\subsection*{Mingzhao Zhao}
Mingzhao was allowed to choose his own area of the project to work on, however, 
not being especially confident with the platform he decided to target user 
interaction with the device via the keypad. As user interaction was a required 
part of the specification it was a useful area of focus. Following suggestions 
of other group members he decided to convert the previous keypad polling method 
used in the mini-projects into an interrupt based system. 

\subsection*{Shivam Mistry} 
Shivam worked very strongly as a group member. To begin with he took on the 
challenge of constructing code to read the data from the CAN bus. Following the 
working CAN bus code, he worked on optimising the CAN bus code through the
implementation of a queue, as well as decoding the CAN packets to extract their
data, and the parsing of text packets to extract only the useful information. 
Furthermore, on realisation of additional device memory, Shivam wrote a memory 
allocator allowing the additional memory to be used constructively, further 
optimising the system. 

\subsection*{Myself (Douglas Parsons)}


\subsection*{Liam Fraser}
Liam has a strong background in music technology, taking the subject at A-level,
and working on personal projects involving music. Furthermore his interest from 
the beginning of the project in audio processing meant that it was an obvious 
choice for Liam to look into the generation and playback of music. The 
specification was exceeded with his contributions, as Liam's interest took him 
into writing additive synthesis for the project, as well as attack sustain release 
curves for the audio to produce a more realistic sounding output, and a fixed 
point maths library in order to replace floating points, hence making his audio 
code more efficient. 

\section{Technical innovation and implementation of each member}
\subsection*{Mingzhao Zhao}
Mingzhao's work was primarily focused on handling the 16 digit keypad on the 
mbed board, this board was configured to be interrupt based as opposed to  being 
polled. This caused additional problems with bounce on the switches, this 
caused multiple interrupts and race conditions to occur within the interrupt 
handler. Due to the row/column method in which the key pad is scanned, the 
number of 'bounces' was reduced significantly by only permitting certain 
input values to register, and then reduced further by only allowing a single key 
press every half a second. This method of debounce was achieved using a system 
timer (SysTick) to count the time between successive key presses, and disregard 
any that occurred too closely together. 

\subsection*{Shivam Mistry}
Shivam initially targeted the CAN bus, and was able to rapidly achieve a fully 
functioning prototype. The original solution processed each packet as it was 
received. As the complexity of songs increased it became obvious that this 
rudimentary method was not going to be sufficient when many CAN packets were 
arriving in a short space of time. In order to improve the functionality the 
packets were instead added to a queue, which was continually processed. This 
removed the requirement of each packet being processed as it was received, and 
allowed a more streamlined solution. Shivam's contribution also included the 
reverse engineering and decoding of the CAN bus packets received. Not only did 
his solution test the check-sum of each packet to ensure that it had been 
correctly received, but also later implemented a text parser, allowing the 
text packets which were sent prior to each song containing useful information 
to be stripped down to only the required information. 

Furthermore, following an 
instance where the physical memory (RAM) on the device was fully used up, Shivam 
discovered that the device contains an additional 32Kb of memory that is 
typically reserved for communicating to a USB or Ethernet device (in addition 
to the original 32Kb). However, due to the inaccessible status of this memory, 
a memory allocator was required for it to be accessed. Therefore, Shivam made a 
memory allocator that allowed access to this additional memory, effectively 
allowing us to store a much larger amount of information on the device. In 
addition, the look-up table that the memory allocator used was stored in the 
additional memory, and hence had no impact otherwise on the functionality of the 
device. 

\subsection*{Myself (Douglas Parsons)}

My personal contribution to the project was primarily targeted at the user 
interactions with the device. While it was initially set out that Mingzhao was 
going to work on implementing an interrupt based keypad, I decided that the 
communication via serial would potentially allow a much greater level of 
debugging, interaction, and setting manipulation for the device. From the 
beginning of the ten week period, I focused on setting up a shell style interface 
for the device. This would allow the user to communicate with the mbed board by 
typing in commands on a computer keyboard, and execute the command by pressing 
the enter key. This therefore allows a much more powerful system of interaction 
with the mbed board than the 16 digit keypad could possibly provide. In addition 
to implementing this additional interfacing method, I focused on customising the 
code from the mini-projects to allow a greater level of control over the on-board 
LCD screen. Implementing features such as scrolling text (of custom messages), 
and separated out writing to each line of the screen. Following the implementation 
of my interfacing method, I worked alongside Mingzhao on combining his code with 
the rest of the project, enabling his input to be a part of the team's solution.

\subsection*{Liam Fraser}
Liam Fraser, following an interest in music technology was eager to work on the 
audio code for the project. He implemented a simplistic system, allowing musical 
notes to be generated by scanning, and interpolating over a look up table. He then 
looked into generating synthesised music, and implemented additional synthesis 
into the project, allowing multiple notes to be playing simultaneously. However, 
on testing his synthesis code on the mbed board, it rapidly became apparent that 
the use of floating point numbers was not sufficiently quick to produce clean 
sounding audio. Therefore, a fixed point library was implemented, this allowed 
much quicker calculations to occur compared to those of floating point numbers. 
Following this optimisation of the synthesis code, the synth code was further 
optimised through the use of Direct Memory Access (DMA). This allowed the audio 
output to work independent of the central processing unit (CPU). This 
therefore speeds up the processing of any audio, as the CPU does not have to be 
involved for any sound to be output, only for the generation of audio samples 
[dma-book]. In addition, to create a more realistic sounding output from the 
device, an attack sustain release state model was incorporated for each note. 
This gives an initial rise in volume as the note is turned on, and a fading out 
of the note as it is turned off, producing a volume curve more typical of an 
actual instrument [asr-book].

