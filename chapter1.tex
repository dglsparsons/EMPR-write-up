% Guidance - 2 pages
% A summary of your project work, experiences as a team, expectations and
% actual outcomes, and some 'feel' for what the rest of your report is about
\section{Summary of project work}
The Second Year Embedded Systems Project was a ten week project, undertaken 
for the duration of Spring Term 2014-2015. The project was a primarily a 
group task, completed in groups consisting of four members. However, there 
was also the capability for individual extensions to the main body of the 
project, allowing each member to showcase individual competency and creativity.

The group project involved programming an ARM LPC1768 'MBED' microcontroller, 
situated on top of a board of peripheral accessories, in order to 
generate and play music on a user selectable channel corresponding to data 
being sent down a Controlled Area Network bus (CAN bus). The individual 
component was not specified, and was instead left up to each individual to 
decide on an extension project, research, and implement it. 

The groups solution was very complete, not only matching the full specification
, but also extending the implementation to a high degree, producing a user 
friendly, good sounding end product. My individual extension provided a more 
user friendly interface than the 
rudimentary keypad attached to the MBED board through the implementation of a 
shell type interface. This shell style interface allows the user to input 
commands via a computer keyboard in order to change modes, adjust settings,
or display useful information.

 
\section{Experiences as a team}
The capability to work successfully as a group was imperative to the triumph of
the project. Throughout the ten week period we had regular group meetings, 
allowing us to keep track of each person's progress, and we had strong levels 
of feedback on contributions to the solution through regular code reviewing 
sessions. This enabled us to remain productive and on track throughout the 
project. 
Furthermore, the high levels of communication enabled each individual to work 
on their preferred areas, or areas of interest without ever straying too far 
away from the desired end goal. 
Each member of the group had a strong drive to 
provide a high quality finished product, and eagerness to develop their 
personal skills. This resulted in a large amount of time being spent in the 
labs outside of the scheduled practicals, enabling us as a group to progress 
much further than we otherwise might have done. Overall I feel we worked very 
well together, and the exercise served as a valuable insight into a fluent 
working environment as a team. 

\section{Expectations and actual outcomes}
At the beginning of the ten week period it was very difficult to predict what 
would be possible to achieve due to a combination of  inexperience with 
programming embedded systems, and an uncertainty regarding the limitations 
of the LPC1768 board used. However, it was expected that during the 
course of the ten weeks my group would be capable of meeting the specifications
of the project, as well as implementing our individual solutions. Certain 
aspects of the project took longer than initial expectations would have 
suggested, for example, it took a significant portion of time to implement the 
various different user control methods present in our final solution. Overall, 
our time management skills worked out very well though, and our completed 
solution lives up to, if not exceeds, the expectations of the group. 

\section{What is in the report}
This report sets out to provide a description of our team solution to the set 
assignment, as well as highlighting areas of individual contribution throughout.
The report will begin with a technical description of the set problem, 
including a discussion of requirements and technical challenges that may be 
faced in the meeting of these requirements. 
Following a description of the problem, the group solution will be discussed. 
The discussion will highlight how each section of the requirements has been met 
by the implemented solution, as well as detailing how the problem has been broken 
down for each individual member, presenting each members implementation and 
technical innovation. 
After a discussion of the group's solution, the appropriate testing strategies 
and methods that were used throughout the project will be examined in detail, 
providing feedback on how they have proved useful, and incorporating a 
discussion on a professional/social/ethical/environmental aspect of the solution.
The report will then focus critically on my individual implementation, 
detailing the technical innovation, implementation, and testing undergone for my 
contribution. 
Finally the report will conclude with a reflective summary of work 
undertaken, considering which aspects of the project went smoothly, which areas
did not go as smoothly, how this might be improved in future, and what lessons 
can be taken away from the completion of the project. 
