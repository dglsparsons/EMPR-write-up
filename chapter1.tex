% Guidance - 2 pages
% A summary of your project work, experiences as a team, expectations and
% actual outcomes, and some 'feel' for what the rest of your report is about
\section{Summary of project work}

The Second Year Embedded Systems Project was a ten week project, undertaken 
for the duration of Spring Term 2014-2015. 
The project was a primarily a group task, completed in groups consisting of four 
members. 
However, there was also the capability for individual extensions to the main 
body of the project, allowing each member to showcase competency and creativity.
\par\bigskip\noindent
The group project involved programming an ARM LPC1768 'MBED' microcontroller, 
situated on top of a board of peripheral accessories, in order to 
generate and play audio on a user selectable channel corresponding to data 
being sent down a Controller Area Network bus (CAN bus) \cite{can-wiki}.
The individual component was not specified, and was instead left up to each 
individual to decide on an extension project, research, and implement it. 
\par\bigskip\noindent
My groups solution was very complete, not only matching the full specification, 
but also extending the implementation to a high degree, producing a user 
friendly end product with a high quality audio output. 
My individual extension provided the device with a more user friendly 
interface than the rudimentary keypad attached to the MBED board through the 
implementation of a shell type interface. 
This interface allows the user to input commands via a computer keyboard in 
order to change modes, adjust settings, or display useful information.

 
\section{Experiences as a team}

The capability to work successfully as a group was imperative to the triumph of
the project. 
Throughout the ten week period we worked effectively as a team through a number 
of different means. 
First of all we held regular group meetings. This allowed us to keep track of 
each group member's progress and enabled each member to feedback any potential 
problems they had encountered, as well as ensuring we were meeting the 
specifications of the project. 
Secondly, each contribution to the group solution was passed through a code 
review prior to inclusion. 
This ensured that each member was only adding high 
quality contributions, and enabled group members to learn from each other.
In addition, each member of the group agreed to use the version control system, 
git, coupled with github. This enabled a much easier collaboration of code 
throughout the project, and made communication between individuals much simpler. 
Through the combination of these methods, my group was able to remain highly 
productive throughout the course of the project.
Furthermore, the high levels of communication enabled each individual to work 
primarily on their areas of interest, removing the need to continually worry 
about to the overall progress of the project. 
Overall I feel we worked very well together as a team, and the project served 
as a valuable insight into a fluent working environment as a team. 

\section{Expectations and actual outcomes}

Due to my own personal inexperience programming embedded systems, and 
unfamiliarity with the LPC1768 board used, it was very difficult to predict the 
outcome of the project. 
However, due to the competency and eagerness of my groups members, it was 
expected from a very early stage that would would rapidly be able to match the 
specification and extend the project to a high degree. 
The majority of the project went very fluently, however, certain aspects of the 
project took much longer than initially expected. 
Notably the CAN bus in the labs was not working at the beginning of the ten week
period, and consequently producing working code for reading the CAN packets took 
longer than expected. 
Furthermore, handling the wide range of possible cases for 
user input, and interaction between the keyboard and shell input methods took much 
longer than expected. 
The end product that we implemented was capable of matching each point of the 
specification, and surpassed the specification within the group project, 
as well as individual extension project. 
Therefore I feel the groups time management skills worked very well, and the end 
product lived up to expectations.

\section{What is in the report}

This report sets out to provide a description of our team solution to the set 
assignment, as well as highlighting areas of individual contribution throughout.
The report will begin with a technical description of the set problem, 
including a discussion of requirements and technical challenges that may be 
faced in the meeting of these requirements. 
Following a description of the problem, the group solution will be discussed. 
The discussion will highlight how each section of the requirements has been met 
by the implemented solution, as well as detailing how the problem has been broken 
down for each individual member, presenting each members implementation and 
technical innovation. 
After a discussion of the group's solution, the appropriate testing strategies 
and methods that were used throughout the project will be examined in detail, 
providing feedback on how they have proved useful, and incorporating a 
discussion on a social aspect of the solution.
The report will then focus critically on my individual implementation, 
detailing the technical innovation, implementation, and testing undergone for my 
contribution. 
Finally the report will conclude with a reflective summary of work 
undertaken, considering which aspects of the project went smoothly, which areas
did not go as smoothly, how this might be improved in future, and what lessons 
can be taken away from the completion of the project. 
