% Guidance - 3 pages
% Description and discussion of your individual component solution, detailing
% the technical innovation and implementation and details of your testing
% strategy, and the results. 
\section{Description of Individual Component Solution}

My individual contribution to the project consisted primarily of a shell style 
interface for the host board, allowing user interaction via a computer keyboard 
and monitor, but also consisted of the modification and extension of previously 
existing LCD screen code, to allow a much greater degree of control over the 
attched LCD display. 

The shell style interface was set up using an interrupt based system from the 
computer. This enables users to enter commands through any input method for 
the computer and process them on the device. This method also allows the user
to display useful information through the computer monitor.

There were a wide range of possible input commands available through this interface, 
with many possible options and settings being adjusted on the fly. By typing 'help' and 
pressing enter, the user could view a list of possible commands, as seen below: 

"List of available commands:\par\bigskip\noindent
playnote note volume   : plays the selected midi note\par\noindent
noteoff note           : turns off any playing notes\par\noindent
volume <vol>           : sets the output volume to 'vol'\par\noindent
showvol                : displays the current volume\par\noindent
write "text"           : writes text to the LCD screen\par\noindent
writeline "text" <linenumber> : writes text to one line of the 
LCD screen\par\noindent
listen                 : listens to music on the CAN bus\par\noindent
stoplisten             : stops listening to music on the CAN bus\par\noindent
setid <channel>        : Filters out channels that do not match 
channel number 'channel'. 
To play all, use "setid all\par\noindent
showid                 : Shows the id of the current channel\par\noindent
showtrack              : Shows the current track name on the LCD\par\noindent
showchan               : Scrolls the channels on the LCD\par\noindent
showpacket             : Prints CAN packets until a key is pressed\par\noindent
scroll \"text\" <line> : displays scrolling text on line <line> of the 
LCD screen\par\noindent
stopscroll <line>      : stops scrolling text on the screen. To stop both lines, 
enter <line> as 2 or all.\par\noindent
scrollenable <line>    : enables scrolling text on the screen.\par\noindent
shownotes              : displays all notes currently being played\par\noindent
cowsay \"text\"        : displays an ASCII cow, saying text\par\noindent
clear <line>           : clears any text that is on the lcd line <line> 
Note - you may also have to call stopscroll to fully clear\par\noindent
\par\bigskip\noindent
This therefore provides the user with a high amount of control over the devices 
output though both audio and visual elements, as well as a high level of control 
over the devices functionality. Furthermore, the interaction with 
the on board LCD display can be clearly seen through the above commands, hence 
explicating the necessity for the alterations made to LCD display code to permit 
a greater degree of control, and enable features such as scrolling text, and 
writing to only part of the LCD screen. 
\section{Discussion of Technical Innovation and Implementation}
\section{Testing Strategy, and the Results}
