% Guidance - 3 pages
% Description and discussion of your individual component solution, detailing
% the technical innovation and implementation and details of your testing
% strategy, and the results. 
\section{Description of Individual Component Solution}

My individual contribution to the project consisted primarily of a shell style 
interface for the host board, allowing user interaction via a computer keyboard 
and monitor. However, my individual contribution also consisted of the 
modification and extension of previously existing LCD screen code in order to 
achieve a much greater degree of control over the attached LCD display. 
\par\bigskip\noindent
The shell style interface was set up using an interrupt based system from the 
computer. This enables users to enter commands into GNU Screen running on 
the computer through any input method. These inputs form commands that are 
then processed and executed on the device to perform tasks such as displaying 
information on the host board, adjusting device settings, or allowing users 
to display useful information through the computer monitor.
\par\bigskip\noindent
There were a wide range of possible input commands available through this 
interface, with many possible options and settings being adjusted on the fly. 
By entering 'help' and pressing enter, the user could view a list of possible 
commands, as seen below: 

\begin{table}[h]
\begin{tabular}{lp{8cm}}
List of available commands:\par\bigskip\noindent & \\
playnote note volume   & : plays the selected midi note\\
noteoff note           & : turns off any playing notes\\
volume vol           & : sets the output volume to 'vol'\\
showvol                & : displays the current volume\\
write text           & : writes text to the LCD screen\\
writeline text linenumber & : writes text to one line of the 
LCD screen\\ 
listen                 & : listens to music on the CAN bus\\
stoplisten             & : stops listening to music on the CAN bus\\
setid channel        & : Filters out channels that do not match 
channel number channel. To play all, use setid all\\
showid                 & : Shows the id of the current channel\\
showtrack              & : Shows the current track name on the LCD\\
showchan               & : Scrolls the channels on the LCD\\
\end{tabular}
\end{table}
% 2 tables because there's a page break in between
\begin{table}[h]
\begin{tabular}{lp{10cm}}
showpacket             & : Prints CAN packets until a key is pressed\\
scroll text line & : displays scrolling text on line line of the 
LCD screen\\
stopscroll line      & : stops scrolling text on the screen. To stop both lines, 
enter line as 2 or all.\\
scrollenable line    & : enables scrolling text on the screen.\\
shownotes              & : displays all notes currently being played\\
cowsay text        & : displays an ASCII cow, saying text\\
clear line           & : clears any text that is on the LCD line line 
Note - you may also have to call stopscroll to fully clear\\
\end{tabular}
\end{table}

\par\bigskip\noindent
The user is therefore provided with a high degree of control over the device's 
functionality. The user interface provided by the shell therefore enables the 
user to display useful information on the computer screen, on the device, 
enter and display custom messages, adjust the devices 
mode of functionality (printing packets, or playing audio), and also gives the 
user control over the audio output. 
Interaction with the onboard LCD display can be clearly seen through the above 
commands, hence explicating the necessity for alterations made to LCD display 
code.
The high degree of control over the LCD additionally enables additional features 
such as scrolling text, or writing multiples times without clearing the LCD 
display. 

\section{Discussion of Technical Innovation and Implementation}

The implementation of the shell style interface and the improvement of the LCD 
display code encountered many different technical challenges throughout. 
The first technical challenge faced was setting up a system enabling user input 
to be transmitted to the device. 
Communication with the MBED board occurs via serial. As text is entered into the 
terminal window of the computer, an interrupt is generated. This input transfers 
the text entered in the terminal into the devices memory for processing at a 
later stage, as well as printing it out on the display so the user can see it. 
Setting up this interrupt provided a significant technical challenge. 
Once this was set up a strong level of communication with the device was enabled. 
A further challenges was then presented with the allocation of memory to store the 
user's input. 
The simplest solution for allocating memory was used: a set input limit is 
defined, and any user input past this limit is not stored in memory, or printed 
to the computer monitor - effectively it is ignored. 
Furthermore, special case characters such as BACKSPACE required additional 
management to remove both the characters displayed on the monitor, and those 
stored in memory. 
\par\bigskip\noindent
When the enter key is pressed (\textbackslash r\textbackslash n received on 
the interrupt), processing of the user's input command begins. 
First the text is split up into tokens for processing. This 
requires the use of a text parser. The text parser implemented uses a state 
based system, with different states corresponding to whether it is interpreting 
a single word, a phrase (group of words encased in quotation marks), or 
white-space. This breaks the user's command into its constituent components, 
which can then be individually evaluated. However, it was found that memory 
allocation problems could occur due to the large range of possible tokens in 
the input. For example, the user could input 'a a a a a a', and this would be 
interpreted as 6 individual tokens, requiring the allocation of six blocks of 
memory. A solution to this problem was found by simply setting a limit on the 
number of tokens that can be generated by a single command. 
\par\bigskip\noindent
Following the dissection of the user's input into components, a string 
comparison tool \texttt{strcmp} was used to determine what the user's input was, 
enabling the corresponding desired behaviour to be executed. In order to 
prevent race conditions from occurring within in the interrupt handler, all text 
processing following the enter key being pressed occurs outside of the interrupt 
handler using a flag based system.
\par\bigskip\noindent
In order to successfully perform the desired functionality, settings are altered 
within many different areas of the project. The implementation of the shell 
therefore required a fully detailed understanding of how each individual 
component of the project functions in order to manipulate and alter existing 
code. Modifications were made throughout the project, and included: the addition 
of a settable ID within the CAN bus code, the addition of a filter in the CAN 
bus to ignore unwanted data according to the preset ID, the addition of a 
volume control in the synth code, and a major overhaul of existing LCD code to 
allow a much greater degree of control over the display. 
\par\bigskip\noindent
The overhaul of the LCD display code provided a significant technical challenge. 
The initial desire was simply to allow scrolling text on the display: however, 
it soon became clear that further alterations were required. Initially the LCD 
code would clear the display prior to every write command. To enable text to be 
written to separate lines of the display independently, or to enable scrolling 
text on one or both of these lines it was therefore necessary to remove this 
behaviour, and discover how to write only to a section of the LCD display. 
This required accessing only certain memory addresses of the LCD display buffer 
in order to prevent clearing parts of the display. Following the successful 
implementation of the capability to write to each individual line of the LCD 
display, the next target was to give the ability to scroll text across the 
display, in order to effectively increase the number of characters that could 
be read. This was more straightforward and was achieved using pointers to 
manipulate the current position in a string, and a system timer (SysTick) to 
update the relevant section of the display at regular intervals.

\section{Testing Strategy, and the Results}

Due to the nature of the extension project, testing was relatively straightforward. 
Using the debugging library, it was very easy to check the effects of any command 
that was entered to ensure correct behaviour. Therefore, any errors or undesirable 
behaviour were rapidly discovered. Furthermore, the interface became a central 
aspect of the group solution, and as a result was used by each member of the 
group on a regular basis to interact with the device. Through each individual's
interaction many test cases were encountered, and therefore my individual extension 
underwent a continual and thorough testing scenario. 
\par\bigskip\noindent
The results of the testing strategy were very positive. The shell style interface 
had a large number of features by the end of the project, and worked very well. 
Furthermore, my individual contribution provided a very valuable tool throughout 
the project due to the high level of interaction that it provided. 
